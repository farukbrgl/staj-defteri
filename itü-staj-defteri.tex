%%%%%%%%%%%%%%%%%%%%%%%%%%%%%%%%%%%%%%%%%%%%%%%%%%%%%%%%%%%%%%%%%%
%%                                          بِسْمِ اللهِ الرَّحْمٰنِ الرَّحِيْمِ %%
%% Rahman ve Rahîm olan Allah'ın adıyla	 						%%
%% In the name of Allah, the Most Gracious, the Most Merciful	%%
%%%%%%%%%%%%%%%%%%%%%%%%%%%%%%%%%%%%%%%%%%%%%%%%%%%%%%%%%%%%%%%%%%

%%%%%%%%%%%%%%%%%%%%%%%%%%%%%%%%%%%%%%%%%%%%%%%%%%%%%%%%%%%%%%%%%%
%% İTÜ Staj Defteri Taslağı
%%
%% LaTeX Template
%%
%% Ömer Faruk Birgül
%% birgulo@itu.edu.tr
%%
%% Bu taslak aşağıdaki adreste bulunan
%% yönetmeliğe göre hazırlanmıştır.
%% http://www.sis.itu.edu.tr/tr/yonetmelik/StajGenelEsaslari.pdf
%%%%%%%%%%%%%%%%%%%%%%%%%%%%%%%%%%%%%%%%%%%%%%%%%%%%%%%%%%%%%%%%%%
%% Bu taslak Texmaker 5.0.3 ile oluşturulmuş ve denenmiştir.
%% Quick Build ve PDFLaTex ile denenmiştir.
%%%%%%%%%%%%%%%%%%%%%%%%%%%%%%%%%%%%%%%%%%%%%%%%%%%%%%%%%%%%%%%%%%

%% Kağıt ve yazı boyutu ayarları
\documentclass[12pt,a4paper]{report}
\special{papersize=210mm,297mm}
%% Kağıt ve yazı boyutu ayarları sonu

%% Türkçe dil paketleri
\usepackage[utf8]{inputenc}
\usepackage[T1]{fontenc}
\usepackage[turkish]{babel}
%% Türkçe dil paketleri sonu

%% Times New Roman font
\usepackage{mathptmx}
%% Metinleri denemek için Lorem Ipsum
\usepackage{lipsum}
%% Sayfa kenarlığı ayarları
\usepackage[top=2cm, bottom=2cm, left=2.5cm, right=2.5cm]{geometry}
%% Tablolar ile ilgili paket
\usepackage{array} 
%% Resim ekleme ile ilgili paketler
\usepackage{graphicx}
\usepackage{subcaption}

%% Programlama dili kodları ile ilgili paketler. Kodları yazdığınız dile göre renklendirme ve girintileme yapar.
%% Raporunuzda kod bulunmuyorsa bu kısımları silebilirsiniz.
\usepackage{listings}
\usepackage{color}
\definecolor{dkgreen}{rgb}{0,0.6,0}
\definecolor{gray}{rgb}{0.5,0.5,0.5}
\definecolor{mauve}{rgb}{0.58,0,0.82}
\lstset{frame=tb,
		language=C, % Dili buradan değiştirebilirsiniz
		aboveskip=3mm,
		belowskip=3mm,
		showstringspaces=false,
		columns=flexible,
		basicstyle={\small\ttfamily},
		numbers=none,
		numberstyle=\tiny\color{gray},
		keywordstyle=\color{blue},
		commentstyle=\color{dkgreen},
		stringstyle=\color{mauve},
		breaklines=true,
		breakatwhitespace=true,
		tabsize=3}
%% Kod paketleri sonu

\begin{document}  

%% PDF için bilgiler. Metinde gözükmez, belge özellklerinde gözükür.
\pdfinfo{   /Author (Ömer Faruk Birgül)
   			/Title  (itu staj defteri)
   			/CreationDate (D:20190913090400)
   			/Subject (Staj)
  			/Keywords (Staj;Taslak;Defter;ITU)
		}
%% PDF için bilgiler sonu

%% İçindekiler sayfası.
%% \addcontentsline{toc}{subsection}{"Buraya Yapılan Uygulama Yazılacak"}
%% Buradaki yazılar gözükecek.
\tableofcontents

%% Sayfa altbilgi ayarları. Gerekmedikçe değiştirmeyin.
\mbox{}
\vfill
\begin{center}
\begin{tabular}{|>{\centering}m{3.6cm}|>{\centering}m{7.4cm}|>{\centering}m{4cm}|}
ÖĞRENCİ&ONAY&FİRMA YETKİLİSİ
\tabularnewline
imza &&imza
\end{tabular}
\end{center}
%% Sayfa altbilgi ayarları sonu.
\newpage
\chapter*{Şirket Hakkında Bilgi}

\paragraph{} Buraya şirket hakkında bilgiler yazılacak.
\paragraph{} Şirket hakkında daha fazla bilgi.
%% Sayfa altbilgi ayarları. Gerekmedikçe değiştirmeyin.
\mbox{}
\vfill
\begin{center}
\begin{tabular}{|>{\centering}m{3.6cm}|>{\centering}m{7.4cm}|>{\centering}m{4cm}|}
ÖĞRENCİ&ONAY&FİRMA YETKİLİSİ
\tabularnewline
imza &&imza
\end{tabular}
\end{center}
%% Sayfa altbilgi ayarları sonu.

%%%%%%%%%%%%%%%%%%%%%%%%%%%%%%%%%%%%%%%%%%%%%%%%%%%%%%%%%%%%%%%%%%%%%%%%%%%%%%%%%%%%%
%% Günlük defter sayfaları. Her gün için bir tane kopyalayıp yeni sayfa oluşturun. %%
%%%%%%%%%%%%%%%%%%%%%%%%%%%%%%%%%%%%%%%%%%%%%%%%%%%%%%%%%%%%%%%%%%%%%%%%%%%%%%%%%%%%%

\newpage %yeni sayfa
 
%% Sayfa üstbilgi ayarları. Tırnak içindeki yerleri kendi bilgilerinizle değiştirin.
%% Eğer yazılar boşluklara sığmazsa uzunluklarla oynayın.
\begin{center}
\begin{tabular}{llr}
\parbox[c]{5cm}{Yapılan Uygulama}
&\parbox[c]{8cm}{"Buraya Yapılan Uygulama Yazılacak"} % Tırnak işareti ile birlikte kendi bilginizle değiştirin.
\addcontentsline{toc}{subsection}{"Buraya Yapılan Uygulama Yazılacak"} % Tırnak işareti ile birlikte kendi bilginizle değiştirin.
&\parbox[c]{3cm}{GG.AA.YYYY}%Tarih
\\
\parbox[c]{5cm}{Uygulamanın Yapıldığı Birim}
&\parbox[c]{8cm}{"Buraya Yapılan Birim Yazılacak"} % Tırnak işareti ile birlikte kendi bilginizle değiştirin.
\end{tabular}
\end{center}
%% Sayfa üstbilgi ayarları sonu.

\paragraph{} Stajda yapılan işler yazılacak.
\paragraph{} Daha fazla iş.

%% Resim ekleme
\shorthandoff{=} % Bu satırı silmeyiniz. Türkçe dili ile \includegraphics arasında sorun çıkıyor.
\begin{figure}[h!]
  \centering
  \begin{subfigure}[b]{0.4\linewidth}
    \includegraphics[width=\linewidth]{black}
    \caption{Siyah logo}
  \end{subfigure}
  \begin{subfigure}[b]{0.4\linewidth}
    \includegraphics[width=\linewidth]{blue}
    \caption{Mavi logo}
  \end{subfigure}
  \caption{İki farklı renkte İTÜ logosu}
  \label{fig:coffee}
\end{figure}
\shorthandoff{=} % Bu satırı silmeyiniz. Türkçe dili ile \includegraphics arasında sorun çıkıyor.
%% Resim ekleme sonu

%% Sayfa altbilgi ayarları. Gerekmedikçe değiştirmeyin.
\mbox{}
\vfill
\begin{center}
\begin{tabular}{|>{\centering}m{3.6cm}|>{\centering}m{7.4cm}|>{\centering}m{4cm}|}
ÖĞRENCİ&ONAY&FİRMA YETKİLİSİ
\tabularnewline
imza&&imza
\end{tabular}
\end{center}
%% Sayfa altbilgi ayarları sonu.

%%%%%%%%%%%%%%%%%%%%%%%%%%%%%%%%%%%%%%%%%%%%%%%%%%%%%%%%%%%%%%%%%%%%%%%%%%%%%%%%%%%%%
%% Günlük defter sayfaları sonu.                                                   %%
%%%%%%%%%%%%%%%%%%%%%%%%%%%%%%%%%%%%%%%%%%%%%%%%%%%%%%%%%%%%%%%%%%%%%%%%%%%%%%%%%%%%%

\end{document}